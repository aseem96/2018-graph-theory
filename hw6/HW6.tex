% --------------------------------------------------------------
% This is all preamble stuff that you don't have to worry about.
% Head down to where it says "Start here"
% --------------------------------------------------------------

\documentclass[12pt]{article}

\usepackage[margin=1in]{geometry}
\usepackage{amsmath,amsthm,amssymb}
\usepackage{tikz}
\usepackage{mathtools}

\usepackage{graphicx}
\graphicspath{ {images/} }

\DeclarePairedDelimiter{\ceil}{\lceil}{\rceil}

\usetikzlibrary{arrows}

\newcommand{\N}{\mathbb{N}}
\newcommand{\Z}{\mathbb{Z}}

\newenvironment{theorem}[2][Theorem]{\begin{trivlist}
\item[\hskip \labelsep {\bfseries #1}\hskip \labelsep {\bfseries #2.}]}{\end{trivlist}}
\newenvironment{lemma}[2][Lemma]{\begin{trivlist}
\item[\hskip \labelsep {\bfseries #1}\hskip \labelsep {\bfseries #2.}]}{\end{trivlist}}
\newenvironment{exercise}[2][Exercise]{\begin{trivlist}
\item[\hskip \labelsep {\bfseries #1}\hskip \labelsep {\bfseries #2.}]}{\end{trivlist}}
\newenvironment{question}[2][Question]{\begin{trivlist}
\item[\hskip \labelsep {\bfseries #1}\hskip \labelsep {\bfseries #2.}]}{\end{trivlist}}
\newenvironment{proposition}[2][Proposition]{\begin{trivlist}
\item[\hskip \labelsep {\bfseries #1}\hskip \labelsep {\bfseries #2.}]}{\end{trivlist}}
\newenvironment{corollary}[2][Corollary]{\begin{trivlist}
\item[\hskip \labelsep {\bfseries #1}\hskip \labelsep {\bfseries #2.}]}{\end{trivlist}}

\begin{document}

% --------------------------------------------------------------
%                         Start here
% --------------------------------------------------------------

%\renewcommand{\qedsymbol}{\filledbox}

\title{Homework 6}%replace X with the appropriate number
\author{Dustin Lambright - dalambri \\ Aseem Raina - araina \\ Bihan Zhang - bzhang28 \\ Anshul Fadnavis - asfadnav\\
%replace with your name
CSC 565 - Graph Theory} %if necessary, replace with your course title

\maketitle


\begin{question}{3}
Show that if $\chi(G) = k$ then $\chi(G^{c}) \geq \ceil[big]{v/k}.$ For every pair $n, k$ with $n \geq k \geq 1$, find a graph $G$ with $ν(G) = n, \chi(G) = k$, and $\chi(G^{c}) = \ceil[big]{v/k}$.
\end{question}

\begin{question}{4}
Prove that $\chi(\bar{H}) = \omega(\bar{H})$ when $H$ is bipartite and has no isolated vertices. (Hint: Phrase the conclusion in terms of H and apply results about bipartite graphs.)
\end{question}

\begin{question}{5}
(!) Prove that Brooks' Theorem is equivalent to the following statement: every $k$-1-regular $k$-critical graph is a complete graph or an odd cycle.

\end{question}





% --------------------------------------------------------------
%     You don't have to mess with anything below this line.
% --------------------------------------------------------------

\end{document}
