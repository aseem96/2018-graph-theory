% --------------------------------------------------------------
% This is all preamble stuff that you don't have to worry about.
% Head down to where it says "Start here"
% --------------------------------------------------------------

\documentclass[12pt]{article}

\usepackage[margin=1in]{geometry}
\usepackage{amsmath,amsthm,amssymb}
\usepackage{tikz}
\usepackage{mathtools}

\usepackage{graphicx}
\graphicspath{ {images/} }

\DeclarePairedDelimiter{\ceil}{\lceil}{\rceil}
\DeclarePairedDelimiter{\floor}{\lfloor}{\rfloor}

\usetikzlibrary{arrows}

\newcommand{\N}{\mathbb{N}}
\newcommand{\Z}{\mathbb{Z}}

\newenvironment{theorem}[2][Theorem]{\begin{trivlist}
\item[\hskip \labelsep {\bfseries #1}\hskip \labelsep {\bfseries #2.}]}{\end{trivlist}}
\newenvironment{lemma}[2][Lemma]{\begin{trivlist}
\item[\hskip \labelsep {\bfseries #1}\hskip \labelsep {\bfseries #2.}]}{\end{trivlist}}
\newenvironment{exercise}[2][Exercise]{\begin{trivlist}
\item[\hskip \labelsep {\bfseries #1}\hskip \labelsep {\bfseries #2.}]}{\end{trivlist}}
\newenvironment{question}[2][Question]{\begin{trivlist}
\item[\hskip \labelsep {\bfseries #1}\hskip \labelsep {\bfseries #2.}]}{\end{trivlist}}
\newenvironment{proposition}[2][Proposition]{\begin{trivlist}
\item[\hskip \labelsep {\bfseries #1}\hskip \labelsep {\bfseries #2.}]}{\end{trivlist}}
\newenvironment{corollary}[2][Corollary]{\begin{trivlist}
\item[\hskip \labelsep {\bfseries #1}\hskip \labelsep {\bfseries #2.}]}{\end{trivlist}}

\begin{document}

% --------------------------------------------------------------
%                         Start here
% --------------------------------------------------------------

%\renewcommand{\qedsymbol}{\filledbox}

\title{Homework 6}%replace X with the appropriate number
\author{Dustin Lambright - dalambri \\ Aseem Raina - araina \\ Bihan Zhang - bzhang28 \\ Anshul Fadnavis - asfadnav\\
%replace with your name
CSC 565 - Graph Theory} %if necessary, replace with your course title

\maketitle


\begin{question}{3}
Show that if $\chi(G) = k$ then $\chi(\bar{G}) \geq \ceil[big]{n/k}.$ For every pair $n, k$ with $n \geq k \geq 1$, find a graph $G$ with $ν(G) = n, \chi(G) = k$, and $\chi(\bar{G}) = \ceil[big]{n/k}$.\\
\\
Let $G$ have a coloring $c$ and $\bar{G}$ have a coloring $c'$\\
The graph $H$ formed by combining $G$ and $\bar(G)$ would then be $K_n$, with $\chi(H)$ = $n$.
\\
Every pair of vertices in $H$ would be adjacent and disjoint in exactly one of $G$ and $\bar{G}$ and therefore every vertex would have a tuple of the form $(c, c')$, with $K_n$ needing at most $size(c) * size(c')$ colours, which would be at least $n$ (since $\chi(H)$ = $n$).\\
In other words,\\
\indent $\chi(G) * \chi(\bar{G}) \geq n$\\
i.e.	$k * \chi(\bar{G}) \geq n$\\
i.e.	$\chi(\bar{G}) \geq \ceil[big]{n/k}$\\
\\
For every pair $n, k$ with $n \geq k \geq 1$, graph $G$ with $ν(G) = n, \chi(G) = k$, and $\chi(\bar{G}) = \ceil[big]{n/k}$ would consist of the following disjoint components:\\
\indent $\floor{n/k}$ cliques of size $k$\\
\indent 1 clique of all the remaining vertices\\
This would ensure that for each clique, all vertices would share the same colour in the complement graph, and therefore $\floor{n/k}$ ($+1$, if $n$ is not exactly divisible by $k$) $= \ceil[big]{n/k}$ colours would be needed, ensuring that $\chi(\bar{G}) = \ceil[big]{n/k}$.

\end{question}

\begin{question}{4}
Prove that $\chi(\bar{H}) = \omega(\bar{H})$ when $H$ is bipartite and has no isolated vertices. (Hint: Phrase the conclusion in terms of H and apply results about bipartite graphs.)\\

Let $X$ and $Y$ be bipartitions of $H$ and let $|X| \geq |Y| $

By Proposition 5.1.7: For every graph $H$, $\chi(H)\geq\omega(H)$. To prove that $\chi(H)= \omega(H)$ we need to demonstrate a method to color $\bar{H}$ using only $\omega(\bar{H})$ colors.\\

Since cliques in $\bar{H}$ are independent sets in $H$, this is the same as coloring $\bar{H}$ with $\omega(\bar{H}) =  \alpha(H)$ colors.\\ 

We know that $\alpha(H) = \beta'(H)$ for a bipartite with no isolated vertices\\

We also know that $\alpha'(H) + \beta'(H) = n(H)$, which can be combined with the previous formula to give $\alpha(H) = n(H)-\alpha'(H)$. \\

Let M be a maximum matching of $\bar{H}$, for every edge $uv$ in M with $u \epsilon X$ and $v \epsilon Y$, assign every $uv$ pair the same color (so ${uv}_1$ gets color 1, ${uv}_x$ gets color x). For every other vertex in $X$ and $Y$ assign a unique color. There are $n-2|M|$ vertices not in M, the number of colors used is $|M| + (n-2|M|) = n - \alpha'(H) = \omega(\bar{H})$ . Thus proving that the inverse of any bipartite graph with no isolated vertices (H) can be colored with $\omega(\bar{H})$ colors.

\end{question}

\begin{question}{5}
(!) Prove that Brooks' Theorem is equivalent to the following statement: every $k-1$-regular $k$-critical graph is a complete graph or an odd cycle.\\

Brooks' Theorem:
If $G$ is a (simple) connected graph other than a clique or an odd cycle, then $\chi(G) = \Delta(G)$.
\end{question}





% --------------------------------------------------------------
%     You don't have to mess with anything below this line.
% --------------------------------------------------------------

\end{document}
