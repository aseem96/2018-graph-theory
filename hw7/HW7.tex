% --------------------------------------------------------------
% This is all preamble stuff that you don't have to worry about.
% Head down to where it says "Start here"
% --------------------------------------------------------------

\documentclass[12pt]{article}

\usepackage[margin=1in]{geometry}
\usepackage{amsmath,amsthm,amssymb}
\usepackage{tikz}
\usepackage{mathtools}
\usepackage{enumitem}

\usepackage{graphicx}
\graphicspath{ {images/} }

\DeclarePairedDelimiter{\ceil}{\lceil}{\rceil}
\DeclarePairedDelimiter{\floor}{\lfloor}{\rfloor}

\usetikzlibrary{arrows}

\newcommand{\N}{\mathbb{N}}
\newcommand{\Z}{\mathbb{Z}}

\newenvironment{theorem}[2][Theorem]{\begin{trivlist}
\item[\hskip \labelsep {\bfseries #1}\hskip \labelsep {\bfseries #2.}]}{\end{trivlist}}
\newenvironment{lemma}[2][Lemma]{\begin{trivlist}
\item[\hskip \labelsep {\bfseries #1}\hskip \labelsep {\bfseries #2.}]}{\end{trivlist}}
\newenvironment{exercise}[2][Exercise]{\begin{trivlist}
\item[\hskip \labelsep {\bfseries #1}\hskip \labelsep {\bfseries #2.}]}{\end{trivlist}}
\newenvironment{question}[2][Question]{\begin{trivlist}
\item[\hskip \labelsep {\bfseries #1}\hskip \labelsep {\bfseries #2.}]}{\end{trivlist}}
\newenvironment{proposition}[2][Proposition]{\begin{trivlist}
\item[\hskip \labelsep {\bfseries #1}\hskip \labelsep {\bfseries #2.}]}{\end{trivlist}}
\newenvironment{corollary}[2][Corollary]{\begin{trivlist}
\item[\hskip \labelsep {\bfseries #1}\hskip \labelsep {\bfseries #2.}]}{\end{trivlist}}

\begin{document}

% --------------------------------------------------------------
%                         Start here
% --------------------------------------------------------------

%\renewcommand{\qedsymbol}{\filledbox}

\title{Homework 7}%replace X with the appropriate number
\author{Dustin Lambright - dalambri \\ Aseem Raina - araina \\ Bihan Zhang - bzhang28 \\ Anshul Fadnavis - asfadnav\\
%replace with your name
CSC 565 - Graph Theory} %if necessary, replace with your course title

\maketitle


\begin{question}{1} - \color{blue} ANSHUL \color{black}
6.1.8  Prove that every simple planar graph has a vertex of degree at most 5.\\
\\
Proof by contradiction:\\
Let us assume that there exists a simple, planar graph $G$ with no vertex of degree $\leq 5$.\\
The simplest case would be that $\delta(G) = 6$\\
According to Theorem <>:\\
\begin{equation}
e \leq 3n - 6
\end{equation}
Also, since $\delta(G) = 6$, we have:\\
\begin{equation}
e \geq 6n
\end{equation}
From (1) and (2):\\\indent$6n \leq 3n - 6$\\
Therefore\\\indent$n \leq -2$\\
Since a graph needs a non-negative number of vertices, this is a contradiction.\\
Hence, our assumption is false.
\end{question}

\begin{question}{2} - \color{blue}BIHAN\color{black} - 
6.1.27  Let $G$ be a connected 3-regular plane graph in which every vertex lies on one face of length 4, one face of length 6, and one face of length 8.
\begin{enumerate}[label=\alph*)]
  \item In terms of $n(G)$, determine the number of faces of each length.
  \item Use Euler's Formula and part (a) to determine the number of faces of $G$.
\end{enumerate}
\end{question}

\begin{question}{3} \color{blue} ASEEM \color{black}
 6.1.29  Prove that the complement of a simple planar graph with at least 11 vertices is nonplanar.  Construct a self-complementary simple planar graph with 8 vertices.
\end{question}

\begin{question}{4} - \color{blue}BIHAN\color{black} - 
6.2.1  Prove that the complement of the 3-dimensional cube $Q_3$ is nonplanar.
\end{question}

\begin{question}{5}  - \color{blue}DUSTIN\color{black} - 
6.2.2 (a,b) Give two proofs that the Petersen graph is nonplanar.
\begin{enumerate}[label=\alph*)]
  \item Using Kuratowski's Theorem.
  \item Using Euler's Formula and the fact that the Petersen graph has girth 5.
\end{enumerate}

\end{question}




% --------------------------------------------------------------
%     You don't have to mess with anything below this line.
% --------------------------------------------------------------

\end{document}
